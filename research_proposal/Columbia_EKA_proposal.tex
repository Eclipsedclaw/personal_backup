%------------------------------------------------------------------------------
% SIMPLE TEMPLATE FOR P1 SCIENTIFIC JUSTIFICATION
%
% This LaTeX2e template is to be compiled with pdflatex.
% Import the resulting pdf file in the P1 interface.
%
% Do not tamper with the settings; changing the font size,
% margins, etc... may result in the proposal not being 
% considered.
% 

\documentclass[11pt,letterpaper]{article}
\usepackage[parfill]{parskip} % Remove paragraph indentation
\usepackage{array} % Required for boldface (\bf and \bfseries) tabular columns
\usepackage{ifthen} % Required for ifthenelse statements
\usepackage{hyperref}

\pagestyle{empty} % Suppress page numbers

% Set margins
\usepackage[a4paper, top=1cm, bottom=1cm, left=0.7cm, right=1.3cm]{geometry}

% Remove font settings for Arial
% \renewcommand{\familydefault}{\sfdefault}
% \usepackage{helvet}
% \renewcommand{\rmdefault}{phv}

\usepackage[utf8]{inputenc}
\usepackage[T1]{fontenc}
\usepackage{anyfontsize}

% Set paragraph spacing
\setlength{\parskip}{5pt}
\setlength{\parindent}{0pt}

% Package for including graphics
\usepackage{graphicx}

% Package for color
\usepackage[dvipsnames]{xcolor}
\definecolor{customblue}{RGB}{0, 122, 204}

% Defines the rSection environment for the large sections within the CV
\newenvironment{rSection}[1]{
  \sectionskip
  {\Large \textbf{\textcolor{Blue}{\MakeUppercase{#1}}}} % Change font size, make it bold, and color it
  \hrule % Horizontal line
  \begin{list}{}{ % List for each individual item in the section
    \setlength{\leftmargin}{1.5em} % Margin within the section
  }
  \item[]
}{
  \end{list}
}


\begin{document}


%------------------------------------------------------------------------------

\begin{rSection}{RESEACH PROPOSAL for Jiancheng(J.C) Zeng}
\vspace{0.25cm}
% science innovation and passion about the opportunity
As an Ernest Kempton Adams Fellow at Columbia University, I intend to pursue an exciting research program in experimental astro-particle physics, building on my research contributions to \textbf{cosmic MeV Gamma-ray observations} and \textbf{indirect dark matter searches}  over the past three years. 

Gamma-ray observations in the unexplored MeV region hold the potential to revolutionize our understanding of some of the Universe’s most extreme astrophysical phenomena. These observations could reveal the powerful relativistic flows from stellar-mass black holes, the activity of supermassive black holes in active galactic nuclei, and the intense behavior of neutron stars, including radio pulsars and magnetars. Additionally, gamma-ray lines from radioactive isotopes provide insights into nucleosynthesis and the processes shaping the Galactic Center, classical Novae, and the cosmic origins of heavy elements. At the same time, indirect dark matter searches have become increasingly popular since the early 2000s. Indirect dark matter (DM) searches take advantage of existing multiwavelength/multi-messenger ensembles of cosmic particle observatories that are designed to study astrophysical phenomena that can test DM properties over a vast range of DM candidate masses. Since indirect searches do not require that DM interact directly inside the detector, they can probe cosmic DM over enormous time/length scales across the Universe and have unique sensitivity to properties such as DM lifetime. 

Recent advancements in detector instrumentation have revitalized interest in MeV gamma-ray science missions. On that front, Columbia has been a leading collaborator on the Gamma-Ray and AntiMatter Survey (GRAMS) experiment~\href{https://www.sciencedirect.com/science/article/pii/S0927650519300805?via%3Dihub#preview-section-abstract}{[1]}, recently recognized as one of NASA's ``Physics of the Cosmos'' missions. GRAMS will use Liquid Argon Time Projection Chamber (LArTPC) technology for the detection of both MeV gamma rays and cosmic charged particles such as antinucleons or antinuclei from DM annihilation, serving both as a gamma-ray telescope and an indirect DM search experiment. With low detection thresholds and excellent energy and spatial resolution, GRAMS promises unprecedented sensitivity into the ``MeV gap'' and competitive sensitivity to DM parameter space. Currently, GRAMS is going through a rapid R$\&$D phase that will culminate into the first GRAMS prototype flight (pGRAMS) planned for fall 2025, followed by an intense design period for the first GRAMS science flight, anticipated as early as 5 years from now. Over the past year, I have been serving as the pGRAMS Subsystem Integration Coordinator and leader of the pGRAMS LArTPC detector design. Building on this experience, as an EKA Fellow I intend to take a leading role in the pGRAMS flight operations and data analysis, as well as in the design of the future GRAMS science mission working closely with Prof.~Georgia Karagiorgi's group on the future GRAMS readout design.  

In recent years, Columbia has also played a leading role in indirect DM searches with the General Antiparticle Spectrometer (GAPS) experiment. I have been an active GAPS collaborator for the past few years, and I intend to continue my involvement in GAPS as an EKA Fellow at Columbia, delivering brand new results on Cosmic-Ray atmospheric detection. I also plan to explore applications of Artificial Intelligence (AI) technology to the upcoming GAPS flight data. Such AI-driven analysis methods will also greatly benefit other future science missions, including GRAMS.

Columbia is deeply integrated with these fields of research, and I am extremely excited for the opportunity to apply for this Fellowship. My expertise in gamma-ray physics and indirect DM searches, and passion and skills for detector instrumentation, would be a perfect fit for Columbia’s ongoing and future projects. Furthermore, the upcoming pGRAMS flight represents the first large Compton camera ever launched in space; I expect that my involvement in this project as an EKA Fellow will position me well for future success and leadership in the field of MeV gamma-ray observations, and solidify my expertise and leading role in indirect DM searches. I am eager to bring my passion to Columbia and look forward to engaging with its vibrant and dynamic community.

\textbf{Proposed Activities:} During my first year as an EKA Fellow, I will deliver the GRAMS prototype balloon flight results, and develop AI-driven data analyses for GAPS. 
For subsequent years, I intend to take a leading role on the future GRAMS science balloon flight design and future satellite mission R$\&$D. %The LArTPC technology offers great science potential for future space missions as it is easily scalable to larger detector volume, providing an easy route to increasing the detector's sensitivity to into the underexplored MeV gap region.
Building upon my AI development efforts during the first year, there will be opportunities to develop and demonstrate selection and reconstruction techniques on pGRAMS simulations and data. For the future GRAMS science mission, I plan to collaborate on cutting-edge electronics and detector development, incorporating AI and machine learning that I will develop and test synergistically on GAPS and pGRAMS data. I am particularly interested in taking the same algorithms and embedding them into the readout electronics onboard the payload. Karagiorgi’s group is leading the development of real-time AI applications for LArTPCs, and this represents an excellent opportunity for collaboration to ensure that future GRAMS missions can extract the maximum physics information from their data. With my background and expertise, I will be a great complement and addition to the Columbia GRAMS team, and I am eager 
learn from their years of experience on LArTPCs as I apply my expertise in DM physics to deliver the first search for indirect DM with GRAMS as my long-term goal.
    
\end{rSection}

% Use same format as CV to make them complete. One year fellowship with 2 additional years depending on cases. Should demonstrate the first year's accomplishment

\end{document}
