\section{Goal of Thesis}
\label{sec:goal}

The goal of this thesis is to explore an uncharted dark matter parameter space of indirect dark matter searches by developing current and next-generation balloon-borne experiments. This thesis aims to address both hardware and analysis challenges in indirect dark matter searches and propose solutions to overcome them. I assembled one of the largest existing indirect dark matter experiment payloads, the General Antiparticle Spectrometer (GAPS) experiment \cite{vondoetinchem2010generalantiparticlespectrometergaps, ARAMAKI201352}, and designed a novel noble-element detector prototype for the next-generation indirect dark matter search experiment, Gamma-Ray and AntiMatter Survey (GRAMS) experiment \cite{aramaki2020dual}. Both of these two experiments have indirect dark matter search capability. Their focus on antideuteron and antihelium3 at $sub-GeV/n$ energy region will allow us to do background-free dark matter searches.

The General Antiparticle Spectrometer (GAPS) is a current-generation mission to measure low-energy cosmic-ray antinuclei during at least three ~35-day Antarctic flights. With its large geometric acceptance and novel exotic-atom-based particle identification, GAPS can detect $\sim$500 cosmic antiprotons per flight and produce a precision cosmic antiproton spectrum in the kinetic energy range of $\sim0.07 - 0.21\ GeV/n$ at the top of the atmosphere \cite{ROGERS2023102791}. GAPS uses 10 layers of lithium-drifted silicon, Si(Li), detectors, provide the necessary X-ray energy resolution of 4 keV for a cost per unit area that is far below that of previously-acquired commercial detectors \cite{PEREZ201812}. By checking the primary vertex and daughter particles generated from the decay of the exotic atom, GAPS is able to identify the incoming charged particles and reconstruct tracks (see Fig.~\ref{GAPS concept}).

\begin{figure}[htbp]
\begin{center} 
\includegraphics*[width=14cm]{images/GAPS_concept_with_payload.png}
\end{center}
\caption{The left picture shows GAPS payload at McMurdo Station. The right figure shows the GAPS detection concept.}
\label{GAPS concept}
\end{figure}


GRAMS is a novel project that can simultaneously target both astrophysical observations with MeV gamma rays and an indirect dark matter search with antimatter detection \cite{aramaki2020dual}. The GRAMS instrument is designed with a cost-effective, large-scale LArTPC (Liquid Argon Time Projection Chamber) detector surrounded by plastic scintillators (see Fig.~\ref{GRAMS concept}). With drifted electrons and a scintillation light trigger, LArTPC is able to reconstruct the primary vertex and perform charged particle detection. Recently, GRAMS was funded to develop and launch a prototype payload, pGRAMS, to verify its detection concept and performance at the 35 km flight altitude. I’m in charge of designing and testing the detector system, including TPC and charge readout, and managing the pGRAMS launch schedule as an onsite coordinator.

\begin{figure}[htbp]
\begin{center} 
\includegraphics*[width=14cm]{images/GRAMS_concept.png}
\end{center}
\caption{GRAMS detection concept}
\label{GRAMS concept}
\end{figure}



