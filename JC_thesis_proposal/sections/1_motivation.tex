\section{Scientific Motivation}
\label{sec:motivation}

The Planck experiment \cite{10.1051/0004-6361/201321591} shows that $ 68\% $ of our universe is composed of dark energy, while $27\%$ consists of dark matter and $5\%$ of baryonic matter. Numerous observations of galaxy rotation have shown that dark matter halo has to exist \cite{10.1046/j.1365-8711.2000.03075.x}. Additionally, recent observations of gravitational lensing in the Bullet Cluster (two colliding clusters of galaxies) indicate the existence of dark matter. However, we still have little understanding of dark matter and how it interacts with baryonic matter. Many theoretical models have been proposed to describe dark matter, such as Weakly Interacting Massive Particles (WIMPs). Various WIMP candidates exist, including neutralinos \cite{donato2000antideuterons, donato2008antideuteron}, right-handed sneutrinos \cite{cerdeno2009right, cerdeno2014low} and right-handed neutrinos in extra dimension theories \cite{baer2005low}. Numerous direct and indirect dark matter search experiments have provided valuable data. Some indirect search experiments with antinuclei are currently in operation or under construction, such as AMS-02, GAPS, and GRAMS \cite{aguilar2002alpha, lubelsmeyer2011upgrade, ajima2000superconducting, hailey2006accelerator, aramaki2020dual}.

Unlike direct dark matter and collider searches, indirect dark matter searches measure the particles from the dark matter annihilation or decay, providing an effective method to test various dark matter models from a different perspective. In particular, indirect dark matter searches with antideuterons or antiheliums have been instrumental in broadening our understanding of the field. Through dark matter self-interactions, various Standard Model particles are expected to be generated. In most cases, the flux is buried in the cosmic background, whereas antideuteron and antihelium-3 fluxes remain distinct due to the natural difficulty of their formation from charged cosmic rays (CRs) interactions. 
%Cosmic-ray propagation is described by the Fokker-Planck equation, which can be written as:

%\begin{equation}
%    \frac{\partial\psi}{\partial t}=Q(\boldsymbol{r}, p)+\boldsymbol{\nabla}\cdot(D_{xx}\boldsymbol{\nabla}\psi-\boldsymbol{V}\psi)+\frac{\partial}{\partial p}p^2D_{pp}\frac{\partial}{\partial p}\frac{\psi}{p^2}-\frac{\partial}{\partial p}\bigg[\psi\frac{dp}{dt}-\frac{p}{3}(\boldsymbol{\nabla}\cdot\boldsymbol{V})\psi\bigg]-\frac{\psi}{\tau}
%\end{equation}

%where $\psi=\psi(\boldsymbol{r}, p, t)$ is the time-dependent CRs density per unit of total particle momentum at position $\boldsymbol{r}$. $Q(\boldsymbol{r}, p)$ is the source term of the CRs, which accounts for both dark matter annihilation sources and secondary background contributions. $D_{xx}, \boldsymbol{V}$, and $D_{pp}$ represent the spatial diffusion coefficient, the convection velocity, and the diffusive re-acceleration coefficient, respectively, commonly referred to as propagation parameters. The last term, $\psi/\tau$, accounts for particle losses due to decay, fragmentation, and inelastic interactions within the Galaxy \cite{PhysRevD.105.083021}.

By projecting cosmic rays (CRs) interacting with the interstellar medium (ISM), we can obtain the expected antideuteron and antihelium-3  background (Secondary) fluxes). Additionally, using a specific decay model, we can also predict the primary antideuteron and antihelium-3 fluxes from dark matter decays (see Fig.~\ref{antideuteron sensitivity},\ref{antihelium3 sensitivity}). From the sub $GeV/n$ energy range, we will be able to detect rare antideuteron or antihelium3 events whose fluxes are two orders of magnitude higher than the background, providing us with background-free signatures.

\begin{figure}[htbp]
\begin{center} 
\includegraphics*[width=7cm]{images/GRAMS_antideuteron.png}
\end{center}
\caption{The antideuteron flux in the $sub-GeV/n$ region exhibits a high signal-to-noise ratio \cite{aramaki2020dual}.}
\label{antideuteron sensitivity}
\end{figure}

\begin{figure}[htbp]
\begin{center} 
\includegraphics*[width=10cm]{images/GRAMS_antihelium3_sensitivity_v5.png}
\end{center}
\caption{The antihelium-3 measurements, similar to the antideuteron measurements, could serve as a powerful tool for background-free dark matter searches \cite{PhysRevD.97.103011, PhysRevD.96.083020, PhysRevD.96.103021, Ding_2019, PhysRevLett.126.101101, PhysRevD.99.023016, PhysRevD.102.063004, Kachelrie_2020}.}
\label{antihelium3 sensitivity}
\end{figure}

