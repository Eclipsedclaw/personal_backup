\section{Related Work}
\label{sec:related-work}

\subsection{Science Simulation Work and Software}
\label{subsec: Science Simulation Work and Software}
\subsubsection{Atmospheric Unfolding}
%\begin{figure}[htbp]
%\begin{center} 
%\includegraphics*[width=12cm]{images/antiproton_atmos.png}
%\end{center}
%\caption{Simulated effect of atmosphere on antiprotons. The left plot shows in general survival probability of antiproton while the right plot shows the effect respect to different zenith angle}
%\label{atmos}
%\end{figure}


For balloon-borne cosmic antinuclei measurement experiments, It is conventional to project the measured particle flux at the balloon location back to the Top of the Atmosphere (TOA) to be compared with theoretical models. To simulate this unfolding, I generated particles using Geant4 simulation to explore the atmospheric effect. I plan to work on providing atmospheric unfolding for the GAPS experiment so we can back project measured particle flux to the TOA flux. In the long term, this work would benefit all experiments that measure cosmic rays in the Earth's atmosphere.

\subsubsection{Antihelium-3 sensitivity for GRAMS Experiment}
The Alpha Magnetic Spectrometer (AMS-02) experiment reported the detection of antihelium-like events \cite{aguilar2013first}, which motivates us to further develop an antihelium-targeted detector that can capture the antiheliums potentially generated by dark matter annihilation (see Fig.~\ref{antihelium3 process}). Since the secondary antihelium-3 flux due to the cosmic-ray interactions would be very small in the low energy ($<1GeV/n$) region, Indicating low-energy antihelium-3 measurements can be background-free dark matter searches.

\begin{figure}[htbp]
\begin{center} 
\includegraphics*[width=8cm]{images/antihelium3_process.png}
\end{center}
\caption{Primary antihelium-3 flux from dark matter self-interation and secondary flux from cosmic-ray interactions with Interstellar Medium}
\label{antihelium3 process}
\end{figure}
To optimize the GRAMS detector design, I used Geant4 simulation to evaluate the detection efficiency and developed the particle identification algorithm for antihelium-3 events. Based on the simulation and the algorithm, I estimated the GRAMS antihelium-3 sensitivity.


\subsubsection{MicroGRAMS Track Reconstruction}
\label{section:track rec}
During the GRAMS R$\&$D phase, I developed several smaller-scale LArTPC detectors to verify the GRAMS detection concept. We constructed a 1/18 size of the pGRAMS flight detector, so called MicroGRAMS, which is identical to one cell of the pGRAMS detector, MiniGRAMS. 
We obtained digitized raw data for 32 charge channels and 4 SiPM channels. I developed analysis tools and algorithms to reconstruct a charged particle primary track, including baseline correction, undershoot compensation, and noise filtration. The details of the hardware setup can be seen in section~\ref{section:tile}. I successfully captured cosmic muon events. And, I'm able to reconstruct incoming muon events using the algorithms that I developed.

%\begin{figure}[htbp]
%\begin{center} 
%\includegraphics*[width=8cm]{images/XY_0635_11050.png}
%\end{center}
%\caption{Raw waveform from CAEN 2740. Top and Bottom plots show two orthogonal directions respectively. Ch31 on both plots represent for light trigger readout by SiPM}
%\label{CAEN raw waveform}
%\end{figure}

%\begin{figure}[htbp]
%\begin{center} 
%\includegraphics*[width=14cm]{images/microGRAMS_track_rec.jpg}
%\end{center}
%\caption{Reconstructed event based on raw CAEN waveform after baseline correction and shaper filtering}
%\label{event rec}
%\end{figure}


\subsection{Hardware Development}
\label{subsec: Hardware Development}
\subsubsection{GAPS Functional Prototype Integration and validation}

We tested small-scale prototype detectors, called GAPS Functional Prototype (GFP), to validate the GAPS detection concept before integrating the full-size science payload. GFP includes 2 x 6 x 4 Silicon detector modules, while the full-sized science detector has 6 x 6 x 10 modules. We also verified the communication and synchronization of the clock and trigger with two sets of Time-of-Flight plastic scintillator panels as an demonstration of the science payload's 20 panels. With GFP, we successfully demonstrated the full function of GAPS with minimal hardware preparation to move forward during the pandemic. During GFP, I especially led the system developments listed below.

\begin{itemize}

\item GFP Thermal System

GAPS uses a novel thermal system for the flight, called Oscillation Heat Pipe (OHP), to cool down the detector \cite{OKAZAKI2021117497}. The OHP system works as a dual-phase self-driven pump; thus, we don't need an active pump for our flight that could introduce vibrations and additional power/weight. GFP used a similar technique with a smaller size of OHP to cool down the detectors to the desired detector performance temperature. I assembled the OHP system and successfully validated both detector performance and thermal system capability (see Fig.~\ref{GFP OHP}).

\begin{figure}[htbp]
\begin{center} 
\includegraphics*[width=10cm]{images/GFP_OHP.png}
\end{center}
\caption{The figure shows the OHP system to cool down the GAPS detectors. With dual-phase coolant inside the system, the heat flows from the bottom of the right detector area to the radiator on the left, as shown in the blue area, where the coolant moves down due to gravity and the heat dissipates. We successfully cooled down all detectors down to $-40^\circ C$.}
\label{GFP OHP}
\end{figure}

%\begin{figure}[htbp]
%\begin{center} 
%\includegraphics*[width=12cm]{images/GFP_detector.png}
%\end{center}
%\caption{Detector perform as good as lab bench test, validated GAPS detection concept as well as thermal system's functionality}
%\label{GFP detector}
%\end{figure}


\item GFP Geometry Simulation
\ \\
Based on my experience building the GFP structure, I implemented the GFP detector geometry to the GEANT4 simulation (see Fig.~\ref{GFP geometry}). We also developed software to analyze the data for the detector testing with GFP. Together with the simulation results, we successfully reconstructed the muon events.

\begin{figure}[htbp]
\begin{center} 
\includegraphics*[width=12cm]{images/GFP_geometry.png}
\end{center}
\caption{The figure shows the OHP system to cool down the GAPS detectors. With dual-phase coolant inside the system, the heat flows from the bottom of the right detector area to the radiator on the left, as shown in the blue area, where the coolant moves down due to gravity and the heat dissipates. We successfully cooled down all detectors down to $-40^\circ C$ during operation}
\label{GFP geometry}
\end{figure}
\end{itemize}


\subsubsection{Thermal System for GAPS First Science flight Payload}
With the expanded full-size OHP system, we plan to operate the GAPS detector around -40 $^\circ C$. The thermal structure is buried deep within the detectors and all other subsystems. It is the start of the project and a schedule-driven item. During the integration, I especially led these two subsystem assemblies.
\begin{itemize}

\item Ground Cooling System
\ \\
It is essential to validate detector performance on the ground before the flight to claim flight readiness, which requires all the detectors to stay below -35$^\circ C$. The OHP system must provide the necessary heat transportation under flight conditions to the 4 m x 2.6 m flight radiator. Therefore, we developed a dedicated ground cooling system (GCS), including active cooling chillers with methanol line manifold to cover the entire radiator fully insulated with 4-inch thick styrofoam to cool down the radiator (see Fig.~\ref{GCS}).

\begin{figure}[htbp]
\begin{center} 
\includegraphics*[width=14cm]{images/GAPS_cooling.png}
\end{center}
\caption{GAPS science payload ground cooling system. The Left shows the schematics of the design, and the right shows what was being assembled and tested in the Nevis lab.}
\label{GCS}
\end{figure}

\item Flight Thermal System
\ \\
GAPS flight thermal system contains a reservoir, a radiator that dissipates heat, an OHP system with 36 loops that pass through 10 layers of 6 x 6 detector modules, and multiple insulation styrofoam designed to close the exposed area of the payload.
During the integration tests, we were able to cool down all GAPS detectors to the desired temperature.

%\begin{figure}[htbp]
%\begin{center} 
%\includegraphics*[width=12cm]{images/GAPS_ASICs_temperature.png}
%\end{center}
%\caption{GAPS detector temperature, testing happened at Nevis lab on Jan 18th, 2024. Edge channels have slightly higher temperature due to convection heat leak of the geometry effect. During science flight, ambient pressure is expected to drop down to 3 torr, thus convection effect could be ignored.}
%\label{ASICs temperature}
%\end{figure}
\end{itemize}


\subsubsection{GRAMS detector design}

\begin{itemize}

\item pGRAMS Charge Sensing Tile Design
\label{section:tile}
\ \\
Conventionally, a TPC detector utilizes anode wires to read the charge signals, but it is challenging to control the vibration of the wires during the flight. To solve this issue, I designed a 2D tile/PCB for drifted electron signals.
The charge collection tile has a geometry of 33~cm x 33~cm with an active area of 30~cm x 30~cm. The tile has 96 strips in each direction (X and Y), spaced 2.6~mm apart and read independently. Due to the cell separator, 90 strips in each direction will be active, and there are 180 charge readout channels. Each strip is made of pixels connected along the given direction, as shown in Fig.~\ref{fig:tilecloseup}. The ionization electrons will induce X and Y strips on the tile, letting us grab the hit location (see section~\ref{section:track rec} for event reconstruction).

%%% add geometry to picture, update the actual tile
\begin{figure}
    \centering
    \includegraphics[scale=0.12]{images/pGRAMS_tile.png}
    \caption{A.pGRAMS tile demonstration  B.Tile detail front side  C.Tile detail back side. D.Tile real picture}
    \label{fig:tilecloseup}
\end{figure}

\item pGRAMS Charge Sensitive Preamp Design and Performance
\ \\
The charge-sensitive pre-amplifier (CSP) will amplify the charge signals before the data acquisition system. We want the amplification system to be as close to the signal collection location, the tile PCB, as possible to reduce the noise picked up in the system. I designed and tested a JFET-cascade charge-sensitive amplifier in the cryogenic temperature to validate the performance \cite{FABRIS1999545}. We have been continuously modifying CSPs for our final flight version. Fig.~\ref{GRAMS CSP} shows the design of the CSP and some performance evaluation test results.

\begin{figure}[htbp]
\begin{center} 
\includegraphics*[width=12cm]{images/GRAMS_CSP.png}
\end{center}
\caption{Left figure shows GRAMS CSP board layout and model, right figure show GRAMS CSP performance evaluation with different conditions}
\label{GRAMS CSP}
\end{figure}

\item pGRAMS TPC Design and evaluation
\ \\
I have been actively designing the assembling the pGRAMS detector, including the mechanical interface and electric field simulation.
%\begin{figure}[htbp]
%\begin{center} 
%\includegraphics*[width=15cm]{images/pGRAMS_TPC.png}
%\end{center}
%\caption{Top left shows pGRAMS TPC model, top right shows potential value alongside the purple line in the bottom right potential figure. Bottom left shows eField vector on one of the cross section.}
%\label{pGRAMS TPC}
%\end{figure}

\end{itemize}